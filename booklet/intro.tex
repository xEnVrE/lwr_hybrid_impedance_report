\section*{Introduction}
In Robotics one of the most interesting task is the manipulation of objects. In general
it can be distinguished between the task of restraining objects, called grasping,
and the task of manipulating objects with fingers called dexterous manipulation.
\par
%% Among the hand-like end-effectors equipped with fingers proposed in the literature
%% there is the Pisa/IIT SoftHand.
The Pisa/IIT Hand \cite{Catalano2014} is one example of end-effector recently coined as soft hands.
It is simple because driven by one motor only
but at the same time robust because it adapts itself to the shape of grasped objects.
\par
Altough very versatile the simplicity of the hand make the grasp of thin objects,
such as a credit card or a sheet of paper, a very hard task
because such objects do not provide enough contact constraints to shape the hand during the grasp execution. %% on the hand
%% for the grasp to take place.
\par
Recent studies \cite{Bonilla2015} propose to face this problem taking into account
and exploiting hand-environment interaction. For example an object placed on a table
can be dragged until it reacheas the edge of the table and sticks out of it.
Then a standard grasp can be performed as usual. However in order to make
this solution workable the problem of making the dragging phase \emph{safe} must be faced.
Using a \emph{pure position} control strategy the \emph{uncontrolled} contact forces and torques
arising between the hand and the object and between the object and the surface of the table
could damage the object or the hand itself.
\par
In this project a \emph{hybrid force position} controller was implemented and tested on the
scenario described above to show that a force feedback solution allows to accomplish the task safely.
\par
The implementation was done using the ROS Control architecture and the robot used
in the experimental setup was a KUKA LWR $4+$ manipulator.

\subsection*{Contents description}
In section one the task of interest is briefly recalled
and a number of reference frames are introduced. They will be useful
to identify the quantities involved in the design of the controllers.
\par
In section two the main controller which realizes the hybrid
force position strategy is designed by choosing appropriate ``robot impedances''
for each degree of freedom within the so-called \emph{Hybrid Impedance (HIC)} framework
proposed by Spong in the 80s.
\par
In section three the synthesis of an inner loop inverse dynamics
controller, which is required by the HIC, is discussed. Also the issues of
controlling a kinematically redundant robot, like the LWR $4$+ in the case
of the task described, in the operational space
are discussed and solved using a \emph{dynamically consistent
generalized inverse} of the Jacobian from Khatib.
\par
In section four the Newton-Euler equations are written for
a rigid body attached to a common force/torque sensor in order to understand
how to obtain the measure of the contact forces and torques from the raw
signal of the sensor. This measure is required as a feedback signal within the
HIC framework. Also a simple least squares like approach is proposed
to estimate the mass of the rigid body, its center of mass and some software
induced offset introduced by the sensor itself.
\par
In section five a joint inverse dynamics controller with inverse
kinematics is developed. This controller is used to move the hand near
the object of interest when the starting configuration of the hand
is too far from the object and the HIC controller can not be used due
to the singularities in the attitude parametrization used in HIC
and due to the joints limits that are not handled in HIC.

\par
In the ending section the experimental setup is presented together with
the results of several experiments.
\newpage
