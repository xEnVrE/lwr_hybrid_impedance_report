\documentclass[tikz,14pt,border=10pt]{standalone}

\usepackage{verbatim}

\usepackage{textcomp}
\usetikzlibrary{shapes,arrows,positioning}
\usetikzlibrary{calc,patterns,decorations.pathmorphing,decorations.markings}
\begin{document}
% Definition of blocks:
\tikzset{%
  block/.style    = {draw, thick, rectangle, minimum height = 4em,
    minimum width = 4em},
  sum/.style      = {draw, circle, node distance = 2cm}, % Adder
  input/.style    = {coordinate}, % Input
  output/.style   = {coordinate}, % Output
  anch/.style   = {coordinate} % Anchor
}
% Defining string as labels of certain blocks.
\newcommand{\suma}{}

\begin{tikzpicture}[auto, thick, node distance=2cm, >=triangle 45]
  \draw
  node[input, name = fdes] {}
  node[sum, right= 1cm of fdes] (sum1) {\suma}
  node[block, right = 1cm of sum1] (ze) {\Large $Z_e$}
  node[sum, right= 1cm of ze] (sum2) {\suma}
  node[input, above= 1cm of sum2, name = fenv] {}
  node[block, below = 1cm of sum2.center] (zm_inverse) {\Large $Z_m^{-1}$}
  node[output, right = 2.5cm of sum2] (F) {}
  node[anch, right = 1.5cm  of sum2](anchor1){}
  ;

  \draw[->, thin](fdes) -- node [pos=0]{\Large $v_{des}$}(sum1);
  \draw[->, thin](sum1) -- node [pos=0.4]{\Large $v$}(ze);
  \draw[->, thin](ze) -- node [pos=0.3]{}(sum2);
  \draw[->, thin](fenv) -- node [pos=0]{\Large $-F_{env}$}(sum2);
  %% \draw[->, thin](sum2) -- node[pos=0.3]{}(ze);
  \draw[->, thin](sum2) -- node[pos = 0.7]{\Large $F$}(F);
  \draw[->, thin](zm_inverse.west) -| node[pos=0.97] {$-$}(sum1);
  \draw[->, thin](anchor1) |- (zm_inverse.east);
\end{tikzpicture}
\end{document}

